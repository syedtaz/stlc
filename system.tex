\documentclass[]{acmart}
\usepackage{amsmath, amssymb}
\usepackage{dsfont}
\usepackage{trfrac}

% Helper commands
\newcommand{\nat}{\texttt{Nat}}
\newcommand{\bool}{\texttt{Bool}}

% =================================

\begin{abstract}
  This paper presents an OCaml implementation of the simply typed lambda calculus
  extended with units, sums and products using the SECD machine.
\end{abstract}

\begin{document}
  \title{SECD Machine Implementation of $\lambda_{\rightarrow}$}

  % Author information

  \author{Tazmilur Saad}
  \affiliation{
    \institution{Independent}
    \city{Jersey City}
    \state{NJ}
    \country{USA}}
  \email{ssaad@colgate.edu}

  % Remove Permission Block
  \setcopyright{rightsretained}
  \makeatletter\@printpermissionfalse\makeatother
  \makeatletter\@printcopyrightfalse\makeatother

  \settopmatter{printacmref=false, printccs=true, printfolios=true}

  \maketitle

  \section{Calculus}

  Contexts are an ordered list of variables equipped with a type.

  \vspace*{-2em}
  \begin{align*}
    \Gamma &\coloneq \varnothing \mid \Gamma, x : \tau
  \end{align*}

  The types in our system includes the unit type, sum types, product types, function types
  and the base types.

  \vspace*{-2em}
  \begin{align*}
    \tau &\coloneq \mathds{1} \mid \tau + \tau \mid \tau \times\tau \mid \tau \rightarrow \tau \mid \nat \mid \bool
  \end{align*}

  The syntax of the system includes variables, abstractions, applications,
  let bindings, if-then-else and members of the base types of natural numbers and
  booleans.

  \vspace*{-2em}
  \begin{align*}
    e &\coloneq x \mid \lambda x : \tau . e \mid e_{1} e_{2} \mid \text{let } x = e_1 \text{ in } e_2 \mid
                \text{ if } e_1 \text{ then } e_2 \text{ else } e_3 \mid \mathds{N} \mid \mathds{B}
  \end{align*}

  The typing judgements are as follows.

  \begin{center}
    $
    \trfrac[~\textsc{Unit}]{}{\Gamma \vdash () : \mathds{1}}
      \qquad
    \trfrac[~\textsc{Nat}]{}{\Gamma \vdash \mathds{N} : \nat}
      \qquad
    \trfrac[~\textsc{Bool}]{}{\Gamma \vdash \mathds{B} : \bool}
    $\\
    \vspace*{1em}

    $\trfrac[~\textsc{Var}]{x : \tau \in \Gamma}{\Gamma \vdash x : \tau}
      \qquad
    \trfrac[~\textsc{Lam}]{\Gamma, x : \tau_1 \vdash e : \tau_2}{\Gamma \vdash \lambda x : \tau_1 . e_{1} \tau_{1} \rightarrow \tau_2}
      \qquad
    \trfrac[~\textsc{App}]{\Gamma \vdash e_1 : \tau_1 \rightarrow \tau_2 \qquad \Gamma \vdash e_{2} : \tau_1}{\Gamma \vdash e_{1} e_{2} : \tau_2}$ \\
    \vspace*{1em}

    $\trfrac[~\textsc{Let}]{\Gamma \vdash e_1 : \tau_1 \qquad \Gamma, x : \tau_1 \vdash e_2 : \tau_2}{\Gamma \vdash \text{ let } x = e_1 \text{ in } e_2 : \tau_2}
      \qquad
    \trfrac[~\textsc{Cond}]{\Gamma \vdash e_1 : \bool \qquad \Gamma \vdash e_2 : \tau \qquad \Gamma \vdash e_3 : \tau}{\Gamma \vdash \text{ if } e_1 \text{ then } e_2 \text{ else } e_3 : \tau}$ \\
    \vspace*{1em}

    $
    \trfrac[~\textsc{Inl}]{\Gamma \vdash e : \tau_1}{\Gamma \vdash \text{ inl } e : \tau_1 + \tau_2}
      \qquad
    \trfrac[~\textsc{Inr}]{\Gamam \vdash e : \tau_2}{\Gamma \vdash \text{ inr } e : \tau_1 + \tau_2}
    $\\
    \vspace*{1em}

    $\trfrac[~\textsc{Pair}]{\Gamma \vdash e_1 : \tau_1 \qquad \Gamma \vdash e_2 : \tau_2}{\Gamma \vdash (e_1, e_2) : \tau_1 \times \tau_2}
      \qquad
    \trfrac[~\textsc{Split}]{\Gamma \vdash e_1 : \tau_1 \times \tau_2 \qquad \Gamma, x : \tau_1, y : \tau_2 \vdash e_2 : \tau_3}{\Gamma \vdash \text{ let } (x, y) = e_1 \text{ in } e_2 : \tau_3}$
  \end{center}

  % The base types in the system inclues the natural numbers and booleans.

  % \vspace*{-2em}
  % \begin{align*}
  %     \mathcal{B}    &\coloneq \nat \mid \bool \\
  %     \nat  &\coloneq \mathbf{Z} \mid \mathbf{S}~ \nat \\
  %     \bool &\coloneq \texttt{true} \mid \texttt{false}
  % \end{align*}

  \section{Implementation}

  Typing contexts are implemented using a locally nameless representation.

\end{document}