\documentclass[nonacm]{acmart}
\usepackage{amsmath, amssymb}
\usepackage{dsfont}
\usepackage{trfrac}
\usepackage[backend=biber,datamodel=acmdatamodel,style=acmnumeric]{biblatex}
\addbibresource{bib.bib}

% Helper commands
\newcommand{\nat}{\texttt{Nat}}
\newcommand{\bool}{\texttt{Bool}}
\newcommand{\caseof}[3]{\text{case } #1 \text{ of } #2 ~$\mid$~ #3}
\newcommand{\caseofnobar}[3]{\text{case } #1 \text{ of } #2 $,$~ #3}
\newcommand{\inl}[1]{\textsc{inl } #1}
\newcommand{\inr}[1]{\textsc{inr } #1}

% =================================

\begin{abstract}
  This work documents an autodidactic implementation of an interpreter for
  a simply typed lambda calculus extended unit, sums and product types. The
  interpreter is based on the SECD machine.
\end{abstract}

\begin{document}
  \title{SECD Machine Implementation of $\lambda_{\rightarrow}$}

  % Author information

  \author{Tazmilur Saad}
  \affiliation{
    \institution{Independent}
    \city{Jersey City}
    \state{NJ}
    \country{USA}}
  \email{ssaad@colgate.edu}

  % Remove Permission Block
  \setcopyright{rightsretained}
  \makeatletter\@printpermissionfalse\makeatother
  \makeatletter\@printcopyrightfalse\makeatother

  \settopmatter{printacmref=false, printccs=true, printfolios=true}

  \maketitle

  \section{Calculus}

  Contexts are an ordered list of variables equipped with a type.

  \vspace*{-2em}
  \begin{align*}
    \Gamma &\coloneq \varnothing \mid \Gamma, x : \tau
  \end{align*}

  The types in our system includes the unit type, sum types, product types, function types
  and the base types.

  \vspace*{-2em}
  \begin{align*}
    \tau &\coloneq \mathds{1} \mid \tau + \tau \mid \tau \times\tau \mid \tau \rightarrow \tau \mid \nat \mid \bool
  \end{align*}

  The syntax of the system includes variables, abstractions, applications,
  let bindings, if-then-else and members of the base types of natural numbers and
  booleans.

  \vspace*{-2em}
  \begin{align*}
    e &\coloneq x \mid () \mid \inl{e} \mid \inr{e} \mid (e_1, e_2) \mid \lambda x : \tau . e \mid e_{1} e_{2} \mid \mathds{N} \mid \mathds{B} \\
      &\mid \text{let } x = e_1 \text{ in } e_2 \mid
              \caseofnobar{e}{\textsc{ inl } x \rightarrow e_2}{\textsc{ inr } y \rightarrow e_2}
  \end{align*}

  The typing judgements and the operational semantics are shown in Figure \ref{fig:stlc}
  and \ref{fig:ops} respectively.

  \begin{figure}
    \centering
    $
    \trfrac[~\textsc{Unit}]{}{\Gamma \vdash () : \mathds{1}}
      \qquad
    \trfrac[~\textsc{Nat}]{}{\Gamma \vdash \mathds{N} : \nat}
      \qquad
    \trfrac[~\textsc{Bool}]{}{\Gamma \vdash \mathds{B} : \bool}
    $\\
    \vspace*{1em}

    $\trfrac[~\textsc{Var}]{x : \tau \in \Gamma}{\Gamma \vdash x : \tau}
    \qquad
  \trfrac[~\textsc{Lam}]{\Gamma, x : \tau_1 \vdash e : \tau_2}{\Gamma \vdash \lambda x : \tau_1 . e_{1} \tau_{1} \rightarrow \tau_2}
    \qquad
  \trfrac[~\textsc{App}]{\Gamma \vdash e_1 : \tau_1 \rightarrow \tau_2 \qquad \Gamma \vdash e_{2} : \tau_1}{\Gamma \vdash e_{1} e_{2} : \tau_2}$ \\
  \vspace*{1em}

$
\trfrac[~\textsc{Inl}]{\Gamma \vdash e : \tau_1}{\Gamma \vdash \text{ inl } e : \tau_1 + \tau_2}
  \qquad
\trfrac[~\textsc{Inr}]{\Gamma \vdash e : \tau_2}{\Gamma \vdash \text{ inr } e : \tau_1 + \tau_2}
$\\ \vspace*{1em}

  $\trfrac[~\textsc{Let}]{\Gamma \vdash e_1 : \tau_1 \qquad \Gamma, x : \tau_1 \vdash e_2 : \tau_2}{\Gamma \vdash \text{ let } x = e_1 \text{ in } e_2 : \tau_2}
  \qquad
  \trfrac[~\textsc{Case}]{\Gamma \vdash e : \tau_1 + \tau_2 \quad \Gamma, x_1 : \tau_1 \vdash y_1 : \tau_3 \quad \Gamma, x_2 : \tau_2 \vdash y_2 : \tau_3}{\Gamma \vdash \text{ case } e \text{ of } \textsc{inl } x_1 \rightarrow y_1 \mid \textsc{ inr } x_2 \rightarrow y_2 : \tau_3}
%   \qquad
% \trfrac[~\textsc{Cond}]{\Gamma \vdash e_1 : \bool \qquad \Gamma \vdash e_2 : \tau \qquad \Gamma \vdash e_3 : \tau}{\Gamma \vdash \text{ if } e_1 \text{ then } e_2 \text{ else } e_3 : \tau}
$ \\ \vspace*{1em}

$\trfrac[~\textsc{Pair}]{\Gamma \vdash e_1 : \tau_1 \qquad \Gamma \vdash e_2 : \tau_2}{\Gamma \vdash (e_1, e_2) : \tau_1 \times \tau_2}
  \qquad
\trfrac[~\textsc{Proj 1}]{\Gamma \vdash (e_{1}, e_{2}) : \tau_{1} \times \tau_{2}}{\Gamma \vdash e_1 : \tau_{1}}
  \qquad
\trfrac[~\textsc{Proj 2}]{\Gamma \vdash (e_{1}, e_{2}) : \tau_{1} \times \tau_{2}}{\Gamma \vdash e_2 : \tau_{2}}$

    \caption{Simply typed lambda calculus}
    \label{fig:stlc}
  \end{figure}

  \begin{figure}

    $
    \trfrac[~\textsc{App 1}]{e_1 \rightarrow {e'}_1}{e_1 e_2 \rightarrow {e'}_1 e_2}
      \qquad
      \trfrac[~\textsc{App 2}]{e_2 \rightarrow {e'}_2}{v_1 e_2 \rightarrow v {e'}_2}
      \qquad
      \trfrac[~\textsc{Let}]{e \rightarrow {e'}}{\text{let } x = e \text{ in } e_2 \rightarrow \text{let } x = {e'} \text{ in } e_2}
    $\\ \vspace*{1em}

    $\trfrac[~\textsc{Proj 1}]{e \rightarrow {e'}}{e.1 \rightarrow {e'}.1}
      \qquad
      \trfrac[~\textsc{Proj 2}]{e \rightarrow {e'}}{e.2 \rightarrow {e'}.2}
      \qquad
      \trfrac[~\textsc{Pair 1}]{e_1 \rightarrow {e'}_{1}}{(e_1, e_2)\rightarrow ({e'}_{1}, e_2)}
    $
    \\ \vspace*{1em}

    $\trfrac[~\textsc{Pair 2}]{e_2 \rightarrow {e'}_{2}}{(v_1, e_2) \rightarrow (v_1, {e'}_{2})}
      \qquad
    \trfrac[~\textsc{Inl}]{e \rightarrow {e'}}{\textsc{inl } e \rightarrow \textsc{ inl } {e'}}
      \qquad
    \trfrac[~\textsc{Inr}]{e \rightarrow {e'}}{\textsc{inr } e \rightarrow \textsc{ inr } {e'}}
    $

    \begin{align*}
      (\lambda x : \tau . e) v \rightarrow [x \mapsto v] e && \textsc{AppAbs} \\
      (v_1, v_2).1 \rightarrow v_1 && \textsc{PairBeta1} \\
      (v_1, v_2).2 \rightarrow v_2 && \textsc{PairBeta2} \\
      \caseof{(\textsc{ inl } v~)}{\textsc{inl } x_1 \Rightarrow e_1}{\textsc{inr } x_2 \Rightarrow t_2} \rightarrow [x_1 \mapsto v]e_1
      && \textsc{CaseInl} \\
      \caseof{(\textsc{ inr } v~)}{\textsc{inl } x_1 \Rightarrow e_1}{\textsc{inr } x_2 \Rightarrow t_2} \rightarrow [x_2 \mapsto v]e_1
      && \textsc{CaseInr}
    \end{align*}

    \caption{Small Step Operational Semantics}
    \label{fig:ops}
  \end{figure}


  \section{Soundness}

  \begin{lemma}[Permutation]
    If $~\Gamma \vdash e : \tau$ and $~\Delta$ is a permutation of $~\Gamma$, then
    $~\Delta \vdash e : \tau$. Moreover, the latter derivation has the same depth
    as the former.
  \end{lemma}

  \begin{proof}
    Note that a typing context $\Gamma = (e_n : \tau_n), (e_{n - 1} : \tau_{n - 1}), \dots,
    (e_1 : \tau_1), (e_0 : \tau_0)$ is a sequence which assigns to each $e_i$ a type $\tau_i$.
    A permutation is a bijection $\Delta : \Gamma \rightarrow \Gamma$. We will use
    $\Delta$ as a bijection and $\Delta$ as a typing context interchangeably.

    Assume $~\Gamma \vdash e : \tau$ and $~\Delta$ is a permutation of $~\Gamma$.
    We proceed by structural induction on the typing derivations.

    [\textsc{Unit, Nat, Bool}]: Let $n$ be the length of $\Gamma$. Since $\Delta$ is a bijection,
    there exists some $j \leq n$ such that $\Delta(e_j : \tau_j) = e : \tau$.
    Therefore, $\Delta \vdash e : \tau$ and the depth does not changes.

    [\textsc{Var}]: If $e : \tau \in \Gamma$, then $e : \tau \in \Delta$ since
    $\Delta$ contains the same elements as $\Gamma$. Therefore, we can apply the
    judgement that $\Delta \vdash e : \tau$. Moreover, the depth does not change.

    [\textsc{Inl, Inr, Pair, Proj 1, Proj 2, App}]: In each of these cases, we can
    assume that the permutation property holds for the antecedent. Therefore,
    we can substitute $\Gamma \vdash e_{i} : \tau_{i}$ for some arbitrary $i$ with
    $\Delta \vdash e_{i} : \tau_{i}$ and straightforwardly apply the judgement
    and notice that the depth does not change.

    [\textsc{Let, Case, Lam}]: Each of these cases contain either $\Gamma$ or
    $\Gamma \vdash e : \tau$ for which we can assume the permutation property.
    All of them extend $\Gamma$ with some term $x_j : \tau_j$ but we can also
    extend $\Delta$ with those terms by adding a mapping from that term to itself.
    Therefore, $\Delta$ remains a permutation of $\Gamma$ and it contains the same
    exact elements. Since we satisfy the assumptions, we can apply the judgements
    to reach the same conclusions.
  \end{proof}

  \begin{lemma}[Weakening]
    If $~\Gamma \vdash e : \tau_1$ and $x \notin dom(\Gamma)$, then $~\Gamma, x : \tau_2
    \vdash e : \tau_1$. Moreover, the latter derivation has the same depth as
    the former.
  \end{lemma}

  \begin{proof}
    We assume that extension of a context does not result in naming conflicts.
    We proceed by structural induction on the typing derivation.

    [\textsc{Unit, Nat, Bool}]: These judgements do not assume anything about
    the context. If we extend the context with a non-existing element, we can
    reach the same conclusions. Moreover, there are no subterms so the derivation
    tree's depth does not change.

    [\textsc{Var}]: If we extend $\Gamma$ with a non-existing element, $x : \tau \in \Gamma$
    still holds and we can apply the judgement.

    [\textsc{Inl, Inr, Pair, Proj 1, Proj 2, App}]: We assume that the weakening
    lemma holds for the antecedent These rules do not extend $\Gamma$, so addition of
    another element allows us to reach the same conclusion.

    [\textsc{Let, Case, Lam}]: We assume the weakening lemma holds for the
    antecedent. Since these rules extend $\Gamma$, there are two cases. If we
    extend $\Gamma$ with a variable that has the same type as the assumptions of
    the judgement, then we get that inference for free. If we extend $\Gamma$ with
    a different type, then the variables in the original assumption will still
    exist modulo renaming and we can still apply the judgement.
  \end{proof}

  \begin{theorem}[Progress]
    If $\cdot \vdash x : \tau$ is a well typed term then $x$ is a value or there
    exists some $y$ such that $x \mapsto y$.
  \end{theorem}

  \begin{proof}
    Admitted.
  \end{proof}

  \begin{theorem}[Preservation]
    If $\cdot \vdash x : \tau$ and $x \mapsto y$, then $\cdot \vdash y : \tau$.
  \end{theorem}

  \begin{proof}
    Admitted.
  \end{proof}


  \section{Acknowledgements}

  Thanks to Anton Lorenzen for his guidance on the project.

  \printbibliography

\end{document}