\documentclass[nonacm]{acmart}
\usepackage{amsmath, amssymb, amsthm}
\usepackage{dsfont}
\usepackage{trfrac}
\usepackage{mathtools}
\usepackage{framed}
\usepackage{xcolor}

% Helper commands
\newcommand{\nat}{\texttt{Nat}}
\newcommand{\bool}{\texttt{Bool}}
\newcommand{\unit}{\texttt{Unit}}
\newcommand{\caseof}[3]{\text{case } #1 \text{ of } #2~\(\mid \)~#3}
\newcommand{\caseofnobar}[3]{\text{case } #1 \text{ of } \{#2,~#3\}}
\newcommand{\inl}[1]{\text{inl } #1}
\newcommand{\inr}[1]{\text{inr } #1}
\newcommand{\jdg}[3]{\trfrac[~\textsc{#1}]{#2}{#3}}
\newcommand{\reference}[1]{\text{ref } #1}

% =================================

\begin{abstract}
  This work documents an autididactic process of extending simply typed lambda calculus
  to System F.
\end{abstract}

\begin{document}
\title{From \(\lambda_{\rightarrow}\) to \(\lambda_2\)}

% Author information

\author{Tazmilur Saad}
\affiliation{
  \institution{Independent}
  \city{Jersey City}
  \state{NJ}
  \country{USA}}
\email{ssaad@colgate.edu}

% Remove Permission Block
\setcopyright{rightsretained}
\makeatletter\@printpermissionfalse\makeatother
\makeatletter\@printcopyrightfalse\makeatother

\settopmatter{printacmref=false, printccs=true, printfolios=true}

\maketitle

\section*{Terminology}

We treat typing contexts as sets. \(\Gamma, x : \tau \) is referred to as
``extending'' the context and the extension is sometimes referred to as the new
binding.

\section{Simply Typed Lambda Calculus}

\begin{figure}
  \begin{framed}
    \begin{align*}
      \Gamma & \Coloneqq \varnothing \mid \Gamma, x : \tau &  & \text{typing context}     \\
      v      & \Coloneqq \lambda x : \tau . e \mid b       &  & \text{values}             \\
      b      & \Coloneqq n                                 &  & \text{base types}         \\
      n      & \Coloneqq 0 \mid 1 \mid \dots               &  & \text{naturals}           \\
      \tau   & \Coloneqq \nat \mid \tau \rightarrow \tau   &  & \text{types}              \\
      e      & \Coloneqq x \mid v \mid e~e                 &  & \text{expressions}        \\
      E      & \Coloneqq [\cdot] \mid E e \mid v E         &  & \text{evaluation context}
    \end{align*}

    \[
      \trfrac[~\textsc{Var}]{x : \tau \in \Gamma}{\Gamma \vdash x : \tau}
      \qquad
      \trfrac[~\textsc{Lam}]{\Gamma, x : \tau_1 \vdash e : \tau_2}{\Gamma \vdash \lambda x : \tau_1 . e : \tau_{1} \rightarrow \tau_2}
      \qquad
      \trfrac[~\textsc{App}]{\Gamma \vdash e_1 : \tau_1 \rightarrow \tau_2 \quad \Gamma \vdash e_{2} : \tau_1}{\Gamma \vdash e_{1} e_{2} : \tau_2}
    \]
    \[
      \trfrac[~\textsc{Nat}]{}{\Gamma \vdash n : \nat}
    \]
  \end{framed}
  \caption{Simply typed lambda calculus}
  \label{fig:stlc}
\end{figure}

\subsection{Structural Properties}

\begin{lemma}[Exchange]
  If \(\Gamma_1 , x_1 : \tau_1 , x_2 : \tau_2 , \Gamma_2 \vdash e : \tau \) then \(
  \Gamma_1 , x_2 : \tau_2 , x_1 : \tau_1 , \Gamma_2 \vdash e : \tau \).
\end{lemma}

\begin{proof}
  Induction on the typing derivation of \(\Gamma_1 , x_1 : \tau_1 , x_2 : \tau_2 , \Gamma_2 \vdash e : \tau \).

  \textsc{[Nat]}: This judgement does not depend on the typing context.

  \textsc{[Var]}: \(x : \tau \) must be an element of \(\Gamma_1\) or \(\Gamma_2\),
  or it is equal to \(x_1\) or \(x_2\). In each of these cases, the antecedent
  is satisfied even if we reorder the elements.

  \textsc{[Lam]}: Assume the inductive hypothesis. Reordering the extensions of
  the typing context does not affect the new binding, and we can derive the judgement.

  \textsc{[App]}: Follows from the inductive hypothesis.
\end{proof}

\begin{lemma}[Weakening]
  If \(\Gamma \vdash e : \tau \) and \( x \notin dom(\Gamma)\), then \(\Gamma, x : \tau_x \vdash e : \tau \).
\end{lemma}

\begin{proof}
  Induction on the typing derivation of \(\Gamma \vdash e : \tau \).

  \textsc{[Nat]}: Does not depend on the typing context.

  \textsc{[Var]}: The binding we add to to the context is not within its domain,
  therefore we can apply the judgement.

  \textsc{[Lam]}: Assume the inductive hypothesis. We extend \(\Gamma\) as part
  of the antecedent --- therefore, the judgement follows modulo renaming of further
  extensions.

  \textsc{[App]}: Follows from the inductive hypothesis.
\end{proof}

\begin{lemma}[Contraction]
  If \(\Gamma_1, x_2 : \tau_1, x_3 : \tau_1, \Gamma_2 \vdash e : \tau_2 \) then \(\Gamma_1, x_1 : \tau_1, \Gamma_2 \vdash e[x_3/x_1][x_2/x_1] : \tau_2 \).
\end{lemma}

\begin{proof}
  {\color{red} QED}.
\end{proof}

\subsection{Progress \& Preservation}

\begin{lemma}[Canonical Forms]
  If $v$ is a value of type \nat, then \(v \in \{0, 1, 2, \dots \} \).
  If $v$ is a value of type \(\tau \rightarrow \tau \), then \(v = \lambda x:\tau.e\).
\end{lemma}

\begin{proof}
  The metavariable \(n\) ranges over the natural numbers, and elements denoted by
  \(n\) are the only elements that can have the type \nat~according to the typing
  rules. The \textsc{Lam} rule is the only rule that implies a value has type \(\tau \rightarrow
  \tau\), and it asserts that the value is a lambda abstraction.
\end{proof}

\textsc{Note}. This can also be proven using the inversion of the typing relation.

\begin{theorem}[Progress]
  If~\(\vdash e : \tau \) then \(e\) is either a value or there exists an \(e'\)
  such that \(e \rightarrow e'\).
\end{theorem}

\begin{proof}
  Induction on the typing derivation of \(\vdash e : \tau \).
  \begin{itemize}
    \item [\textsc{Var.}] A variable is not well typed in the empty typing context.
    \item [\textsc{Nat, Lam.}] A natural number and a lambda abstraction are values.
    \item [\textsc{App.}] Let's assume the inductive hypothesis. If either \(e_1\) or
          \(e_2\) are not values, then there exists an \(e'\) such that a reduction step can
          be taken. If they are both values, then the canonical forms lemma implies that
          \(e_1\) is a lambda abstraction since \(e_1 : \tau \rightarrow \tau \), and we can \(\beta \)-reduce \(e_1 e_2\) via
          \(e_1[e_2/x]\).
  \end{itemize}
\end{proof}

\begin{lemma}[Preservation of Types under Substitution]
  If \(\Gamma, x : \tau_1 \vdash e_1 : \tau \) and \(\Gamma \vdash e_2 : \tau_1\)
  then \(\Gamma \vdash e_1[e_2/x] : \tau \).
\end{lemma}

\begin{proof}
  Induction on the typing derivation of \(\Gamma, x : \tau_1 \vdash e_1 : \tau \).
\end{proof}

\section{Extensions}

In each of the following extensions we assume the base system is the simply
typed lambda calculus from Fig~\cite{fig:stlc} unless indicated otherwise.

\subsection{Let}

\begin{figure}
  \begin{framed}
    \begin{align*}
      e & \Coloneqq \text{let } x = e \text{ in } e \mid \dots                                      &  & \text{expressions}        \\
      E & \Coloneqq \text{let } x = E \text{ in } e \mid \text{let } x = v \text{ in } E \mid \dots &  & \text{evaluation context}
    \end{align*}
    \[
      \text{let } x = v \text{ in } e \rightarrow e[v/x]
      \qquad
      \trfrac[~\textsc{Let}]{\Gamma \vdash e_1 : \tau_1 \quad \Gamma, x : \tau_1 \vdash e_2 : \tau_2}{\Gamma \vdash \text{let } x = e_1 \text{ in } e_2 : \tau_2}
    \]
  \end{framed}
  \caption{Let}\label{fig:let}
\end{figure}

\subsection{Unit}

\begin{figure}
  \begin{framed}
    \begin{align*}
      e    & \Coloneqq () \mid \dots    &  & \text{expressions} \\
      v    & \Coloneqq () \mid \dots    &  & \text{values}      \\
      \tau & \Coloneqq \unit \mid \dots &  & \text{types}
    \end{align*}
    \[
      \trfrac[~\textsc{Unit}]{}{\Gamma \vdash () : \unit}
    \]
  \end{framed}
  \caption{Unit}\label{fig:unit}
\end{figure}

\subsection{Tuples}

\begin{figure}
  \begin{framed}
    \begin{align*}
      v    & \Coloneqq (v_0, \dots, v_n) \mid \dots             &  & \text{values}             \\
      \tau & \Coloneqq (\tau_0, \dots, \tau_n) \mid \dots       &  & \text{types}              \\
      e    & \Coloneqq (e_0, \dots, e_n) \mid e.i \mid \dots    &  & \text{expressions}        \\
      E    & \Coloneqq (E, \dots) \mid (v, E, \dots) \mid \dots &  & \text{evaluation context}
    \end{align*}

    \[
      (v_0, \dots, v_n).i \rightarrow v_i
      \qquad
      \trfrac[~\textsc{Proj}]{\Gamma \vdash e : (\tau_0, \dots, \tau_n) \quad i \in \{0, \dots, n\}}{\Gamma \vdash e.i : \tau_i}
    \]

    \[
      \trfrac[~\textsc{Tuple}]{\forall i \in \{0, \dots, n\} \quad \Gamma \vdash e_i : \tau_i}{\Gamma \vdash (e_0, \dots, e_n) : (\tau_0, \dots, \tau_n)}
    \]
  \end{framed}
  \caption{Tuples}\label{fig:tuples}
\end{figure}

\subsection{Records}

Let \(l\) be a metavariable ranging over labels.

\begin{figure}
  \begin{framed}
    \begin{align*}
      v    & \Coloneqq \{l_0 = v_0, \dots, l_n = v_n\} \mid \dots                                              &  & \text{values}             \\
      \tau & \Coloneqq \{l_0 : \tau_0, \dots, l_n : \tau_n \} \mid \dots                                       &  & \text{types}              \\
      e    & \Coloneqq \{l_0 = e_0, \dots, l_n = e_n\} \mid e.l \mid \dots                                     &  & \text{expressions}        \\
      E    & \Coloneqq  \{l_0 = E, \dots, l_n = e_n \} \mid \{l_0 = v_0, \dots, l_i = e_i, \dots \} \mid \dots &  & \text{expression context}
    \end{align*}

    \[
      \{l_0 = v_0, \dots, l_n = v_n\}.l_i \rightarrow v_i
      \qquad
      \trfrac[~\textsc{Proj}]{e : (l_0 : \tau_0, \dots, l_n : \tau_n) \quad i \in \{0, \dots, n\}}{\Gamma \vdash e.l_i : \tau_i}
    \]

    \[
      \trfrac[~\textsc{Tuple}]{\forall i \in \{0, \dots, n\} \quad \Gamma \vdash e_i : \tau_i}{\Gamma \vdash (l_0 = e_0, \dots, l_n = e_n) : (l_0 : \tau_0, \dots, l_n : \tau_n)}
    \]
  \end{framed}
  \caption{Records}\label{fig:records}
\end{figure}

\subsection{Variants}

\begin{figure}
  \begin{framed}
    \begin{align*}
      e & \Coloneqq \langle l = e \rangle~as~\tau \mid \text{case } e \text{ of } \langle l_i = x_i \rangle \Rightarrow e_i \mid \dots && \text{expressions} \\
      \tau & \Coloneqq \langle l_0 : \tau_0, \dots, l_n : \tau_n \rangle \mid \dots && \text{types} \\
    \end{align*}
    \[
      \trfrac[~\textsc{Variant}]{\Gamma \vdash e_j : \tau_j}{
        \Gamma \vdash \langle l_j = e_j \rangle~as~\langle l_0 : \tau_0, \dots, l_n : \tau_n \rangle : \langle l_0 : \tau_0, \dots, l_n : \tau_n \rangle
      }
    \]
  \end{framed}
  \caption{Variants}\label{fig:variants}
\end{figure}

\subsection{Fixpoints}

\begin{figure}
  \begin{framed}
  \end{framed}
  \caption{Fixpoints}\label{fig:fixpoints}
\end{figure}

\subsection{References}

We need the simply typed lambda calculus extended with unit types as in Fig~\ref{fig:unit}.
\(l\) is a metavariable that ranges over memory locations.

\begin{figure}
  \begin{framed}
    \begin{align*}
      e      & \Coloneqq \reference{e} \mid \texttt{!}e \mid e := e \mid \dots &  & \text{expressions}  \\
      v      & \Coloneqq l \mid \dots                                          &  & \text{values}       \\
      \tau   & \Coloneqq \texttt{Ref}~\tau                                     &  & \text{types}        \\
      \mu    & \Coloneqq \varnothing \mid \mu, l = v                           &  & \text{stores}       \\
      \Sigma & \Coloneqq \varnothing \mid \Sigma, l : \tau                     &  & \text{store typing}
    \end{align*}
    \[
      \trfrac[~\textsc{Loc}]{\Sigma(l) = \tau}{\Gamma \mid \Sigma \vdash l : \texttt{Ref}~\tau}
    \]
    \begin{align*}
      l := v \mid \mu \rightarrow () \mid \mu[l \rightarrow v]
      \qquad &
      \trfrac[~\textsc{Assign}]{\Gamma \mid \Sigma \vdash e_1 :
        \texttt{Ref}~\tau \qquad \Gamma \mid \Sigma \vdash e_2 : \tau}{
      \Gamma \mid \Sigma \vdash e_1 := e_2 : \unit}                                                                           \\
      \trfrac[]{\mu (l) = v}{\texttt{!} l \mid \mu \rightarrow v \mid \mu}
      \qquad &
      \trfrac[~\textsc{Deref}]{\Gamma \mid \Sigma \vdash e : \texttt{Ref}~\tau}{\Gamma \mid \Sigma \vdash \texttt{!}e : \tau} \\
      \trfrac[]{l \notin dom(\mu)}{\reference{v} \mid \mu \rightarrow l \mid (\mu, l \rightarrow v)}
      \qquad &
      \trfrac[~\textsc{Ref}]{\Gamma \mid \Sigma \vdash e : \tau}{\Gamma \mid \Sigma \vdash \reference{e} : \texttt{Ref}~\tau}
    \end{align*}
  \end{framed}
  \caption{References}\label{fig:references}
\end{figure}

\subsection{Recursive Types}

\begin{figure}
  \begin{framed}
  \end{framed}
  \caption{Recursive Types}\label{fig:recursive-types}
\end{figure}

\section{System F}

\subsection{Type Reconstruction}

\begin{figure}
  \begin{framed}
  \end{framed}
  \caption{Type Reconstruction}\label{fig:type-reconstruction}
\end{figure}

\subsection{Universals}

\begin{figure}
  \begin{framed}
  \end{framed}
  \caption{Universals}\label{fig:universals}
\end{figure}

\subsection{Existentials}

\begin{figure}
  \begin{framed}
  \end{framed}
  \caption{Existentials}\label{fig:existentials}
\end{figure}

\section{Acknowledgements}

Thanks to Anton Lorenzen for his guidance on the project.

\end{document}