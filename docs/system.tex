\documentclass[nonacm]{acmart}
\usepackage{amsmath, amssymb}
\usepackage{dsfont}
\usepackage{trfrac}
\usepackage{mathtools}
\usepackage[backend=biber,datamodel=acmdatamodel,style=acmnumeric]{biblatex}
\addbibresource{bib.bib}

% Helper commands
\newcommand{\nat}{\texttt{Nat}}
\newcommand{\bool}{\texttt{Bool}}
\newcommand{\caseof}[3]{\text{case } #1 \text{ of } #2~\(\mid \)~#3}
\newcommand{\caseofnobar}[3]{\text{case } #1 \text{ of } \{#2,~#3\}}
\newcommand{\inl}[1]{\text{inl } #1}
\newcommand{\inr}[1]{\text{inr } #1}
\newcommand{\jdg}[3]{\trfrac[~\textsc{#1}]{#2}{#3}}

% =================================

\begin{abstract}
  This work documents an autodidactic implementation of an interpreter for
  a simply typed lambda calculus extended unit, sums and product types. The
  interpreter is based on the SECD machine.
\end{abstract}

\begin{document}
\title{SECD Machine Implementation of \(\lambda_{\rightarrow}\)}

% Author information

\author{Tazmilur Saad}
\affiliation{
  \institution{Independent}
  \city{Jersey City}
  \state{NJ}
  \country{USA}}
\email{ssaad@colgate.edu}

% Remove Permission Block
\setcopyright{rightsretained}
\makeatletter\@printpermissionfalse\makeatother
\makeatletter\@printcopyrightfalse\makeatother

\settopmatter{printacmref=false, printccs=true, printfolios=true}

\maketitle

\section{Calculus}

A typing context is a set of variables equipped with a type.

\vspace*{-2em}
\begin{align*}
  \Gamma & \Coloneqq \varnothing \mid \Gamma, x : \tau
\end{align*}

The types in our system includes the unit type, sum types, product types, function types
and the base types.

\vspace*{-2em}
\begin{align*}
  \tau & \Coloneqq \mathds{1} \mid \tau + \tau \mid \tau \times\tau \mid \tau \rightarrow \tau \mid \nat \mid \bool
\end{align*}

The values in our language include lambda abstractions, unit, sums, products, natural numbers and booleans.

\vspace*{-2em}
\begin{align*}
  v & \Coloneqq \lambda x : \tau . e \mid () \mid (v, v) \mid \inl{v} \mid \inr{v} \mid n \mid b \\
  n & \Coloneqq 0 \mid 1 \mid 2 \mid \dots                                                       \\
  b & \Coloneqq true \mid false
\end{align*}

Expressions includes variables, values, sums, products, applications, let bindings, cases and projections

\vspace*{-2em}
\begin{align*}
  e & \Coloneqq x \mid v \mid \inl{e} \mid \inr{e} \mid (e, e) \mid (e, e).1 \mid (e, e).2 \mid e e           \\
    & \mid \text{let } x = e \text{ in } e \mid \caseofnobar{e}{\inl{x} \rightarrow e}{\inr{x} \rightarrow e}
\end{align*}

Expressions are evaluated using the following evaluation contexts.

\vspace*{-2em}
\begin{align*}
  E & \Coloneqq [\cdot] \mid E e \mid v E \mid \inl{E} \mid \inr{E} \mid (E,e) \mid (v, E) \mid E.1 \mid E.2  \\
    & \mid \caseofnobar{E}{\inl{x} \rightarrow e}{\inr{x} \rightarrow e} \mid \text{let } x = E \text{ in } e
\end{align*}

The typing judgements and the operational semantics are shown in Figure~\ref{fig:stlc}
and~\ref{fig:ops} respectively.

%% ============================================================================
%% Typing Derivations START
%% ============================================================================

\begin{figure}
  \centering
  \(
  \trfrac[~\textsc{Unit}]{}{\Gamma \vdash () : \mathds{1}}
  \qquad
  \trfrac[~\textsc{Nat}]{}{\Gamma \vdash n : \nat}
  \qquad
  \trfrac[~\textsc{Bool}]{}{\Gamma \vdash b : \bool}
  \)
  \\
  \vspace*{1em}

  \(\trfrac[~\textsc{Var}]{x : \tau \in \Gamma}{\Gamma \vdash x : \tau}
  \qquad
  \trfrac[~\textsc{Lam}]{\Gamma, x : \tau_1 \vdash e : \tau_2}{\Gamma \vdash \lambda x : \tau_1 . e : \tau_{1} \rightarrow \tau_2}
  \qquad
  \trfrac[~\textsc{App}]{\Gamma \vdash e_1 : \tau_1 \rightarrow \tau_2 \qquad \Gamma \vdash e_{2} : \tau_1}{\Gamma \vdash e_{1} e_{2} : \tau_2} \) \\
  \vspace*{1em}

  \(
  \trfrac[~\textsc{Inl}]{\Gamma \vdash e : \tau_1}{\Gamma \vdash \text{ inl } e : \tau_1 + \tau_2}
  \qquad
  \trfrac[~\textsc{Inr}]{\Gamma \vdash e : \tau_2}{\Gamma \vdash \text{ inr } e : \tau_1 + \tau_2}
  \) \\ \vspace*{1em}

  \(\trfrac[~\textsc{Let}]{\Gamma \vdash e_1 : \tau_1 \qquad \Gamma, x : \tau_1 \vdash e_2 : \tau_2}{\Gamma \vdash \text{ let } x = e_1 \text{ in } e_2 : \tau_2}
  \qquad
  \trfrac[~\textsc{Case}]{\Gamma \vdash e : \tau_1 + \tau_2 \quad \Gamma, x_1 : \tau_1 \vdash e_1 : \tau \quad \Gamma, x_2 : \tau_2 \vdash e_2 : \tau}{\Gamma \vdash \caseofnobar{e}{\inl{x_1} \rightarrow e_1}{\inr{x_2} \rightarrow e_2} : \tau}
  \) \\ \vspace*{1em}

  \(\trfrac[~\textsc{Pair}]{\Gamma \vdash e_1 : \tau_1 \qquad \Gamma \vdash e_2 : \tau_2}{\Gamma \vdash (e_1, e_2) : \tau_1 \times \tau_2}
  \qquad
  \trfrac[~\textsc{Proj 1}]{\Gamma \vdash (e_{1}, e_{2}) : \tau_{1} \times \tau_{2}}{\Gamma \vdash e_1 : \tau_{1}}
  \qquad
  \trfrac[~\textsc{Proj 2}]{\Gamma \vdash (e_{1}, e_{2}) : \tau_{1} \times \tau_{2}}{\Gamma \vdash e_2 : \tau_{2}} \)

  \caption{Simply typed lambda calculus}\label{fig:stlc}
\end{figure}

%% ============================================================================
%% Typing Derivations END
%% ============================================================================

%% ============================================================================
%% Operational Semantics START
%% ============================================================================

\begin{figure}

  \(
  \jdg{context}{e \rightarrow e'}{E[e] \rightarrow E[e']}
  \) \\ \vspace*{1em}

  \begin{align*}
    (\lambda x : \tau . e)v \rightarrow e[v/x]         &  & \beta\textsc{-reduction} \\
    \text{let } x = v \text{ in } e \rightarrow e[v/x] &  & \textsc{Let}             \\
    (v_1, v_2).1 \rightarrow v_1                       &  & \textsc{Proj 1}          \\
    (v_1, v_2).2 \rightarrow v_2                       &  & \textsc{Proj 2}          \\
    \caseofnobar{\inl{v}}{e_1}{e_2} \rightarrow e_1    &  & \textsc{Inl}             \\
    \caseofnobar{\inr{v}}{e_1}{e_2} \rightarrow e_2    &  & \textsc{Inr}
  \end{align*}

  \caption{Operational Semantics}\label{fig:ops}
\end{figure}

%% ============================================================================
%% Operational Semantics END
%% ============================================================================

%% ============================================================================
%% Substitution START
%% ============================================================================

\begin{figure}
  \centering
  \(
    b[v/x] = b \qquad
    n[v/x] = n \qquad
    ()[v/x] = () \qquad
    (e e)[v/x] = e[v/x] e[v/x]
  \) \\ \vspace*{1em}

  \(
    x_1[v/x_1] &= v \qquad
    x_2[v/x_1] &= x_2 \qquad
    (\inl{e})[v/x] &= \inl{e[v/x]} \qquad
    (\inl{r})[v/x] &= \inr{e[v/x]}
  \) \\ \vspace*{1em}

  \(
    (e, e)[v/x] &= (e[v/x], e[v/x])
  \)

  \caption{Substitution Rules}\label{fig:ops}
\end{figure}

%% ============================================================================
%% Substitution END
%% ============================================================================

\section{Soundness}

\subsection{Preservation}

%% ============================================================================
%% Permutation PROOF
%% ============================================================================

\begin{lemma}[Permutation]
  If \(~\Gamma \vdash e : \tau \) and \(~\Delta \) is a permutation of \(~\Gamma \), then
  \(~\Delta \vdash e : \tau \). Moreover, the latter derivation has the same depth
  as the former.
\end{lemma}

\begin{proof}
  A permutation is a bijection \(\Delta : \Gamma \rightarrow \Gamma \). We will use
  $\Delta$ as a bijection and $\Delta$ as a typing context interchangeably.
  We proceed by structural induction on the typing derivations.

    [\textsc{Unit, Nat, Bool}]: Since $\Delta$ is a bijection, there exists some
  $e' : \tau'$ in $\Delta$ such that \( \Delta(e' : \tau') = e : \tau \). Therefore,
  \(\Delta \vdash e : \tau \) and the depth does not change.

  [\textsc{Var}]: If \(e : \tau \in \Gamma \), then \(e : \tau \in \Delta \) since $\Delta$ is a bijection.
  We can apply the judgement and the depth does not change.

  [\textsc{Inl, Inr, Pair, Proj 1, Proj 2, App}]: We assume the inductive hypothesis, substitute
  $\Delta \vdash e : \tau$ for \( \Gamma \vdash e : \tau \) and apply the judgement.
  The depth does not change.

  [\textsc{Let, Case, Lam}]:  We assume the inductive hypothesis
  and substitute $\Delta \vdash e : \tau$ for \( \Gamma \vdash e : \tau \). Each
  of these derivations extend $\Gamma$ with some term $x : \tau_x$, but we can
  also extend $\Delta$ by adding a mapping from this term to itself and apply the judgement.
  The depth does not change.
\end{proof}

%% ============================================================================
%% Weakening PROOF
%% ============================================================================

\begin{lemma}[Weakening]
  If $~\Gamma \vdash e : \tau_1$ and $x \notin dom(\Gamma)$, then $~\Gamma, x : \tau_2
    \vdash e : \tau_1$. Moreover, the latter derivation has the same depth as
  the former.
\end{lemma}

\begin{proof}
  We proceed by structural induction on the typing derivation.

  [\textsc{Unit, Nat, Bool}]: These judgements do not assume anything about
  the context. If we extend the context with an element from outside the domain, we can
  reach the same conclusions. Moreover, there are no subterms so the derivation
  tree's depth does not change.

  [\textsc{Var}]: If we extend $\Gamma$ with an element from outside the domain, $x : \tau \in \Gamma$
  still holds and we can apply the judgement.

  [\textsc{Inl, Inr, Pair, Proj 1, Proj 2, App}]: We assume the inductive hypothesis.
  These rules do not extend $\Gamma$, so addition of another element from outside the domain allows us to reach the same conclusion.

  [\textsc{Let, Case, Lam}]: We assume the inductive hypothesis. If we extend the
  $\Gamma$ with element from outside the domain, the subsequent extensions as part
  of the antecedent will still be valid modulo renaming. We can still apply the judgement.
\end{proof}

%% ============================================================================
%% Substitution PROOF
%% ============================================================================

\begin{lemma}[Substitution]
  If~\( \Gamma, x : \tau_1 \vdash e : \tau_2 \) and \( \Gamma \vdash v : \tau_1 \) then \( e[v/x] : \tau_2 \).
\end{lemma}

\begin{proof}
  We proceed by structural induction on the typing derivation of \(\Gamma, x : \tau_1 \vdash e : \tau_2\).

  [\textsc{Unit, Nat, Bool}]: By definition of substitution.

  [\textsc{Var}]:

  \vspace*{-1em}
  \begin{align*}
    & \trfrac[]{z : \tau_1 \in \Gamma, x : \tau}{\Gamma, x : \tau \vdash z : \tau_1} \quad \Gamma \vdash v : \tau && \text{(Assumptions)} \\
    & x = z \implies z : \tau \in \Gamma \implies \Gamma \vdash z[v/x] : \tau = \tau_1 && \text{(Substitution)} \\
    & x \neq z \implies z : \tau_1 \in \Gamma \implies \Gamma \vdash z[v/x] : \tau_1 && \text{(Substitution)}
  \end{align*}

  [\textsc{Lam}]:

  \vspace*{-1em}
  \begin{align*}
    & \trfrac[]{\Gamma, x : \tau, z : \tau_1 \vdash e : \tau_2}{\Gamma, x : \tau \vdash \lambda z : \tau_1 . e : \tau_{1} \rightarrow \tau_2}
        \quad \Gamma \vdash v : \tau  && \text{(Assumption)} \\
    & \Gamma, x : \tau, z : \tau_1 \vdash e : \tau_2 && \text{(Subterm)} \\
    \implies & \big(\Gamma, z : \tau_1), x : \tau \vdash e : \tau_2 && \text{(Permutation)} \\
    \implies & \Gamma \vdash e[v/z] : \tau_2 && \text{(Inductive Hypothesis)} \\
    & \Gamma \vdash v : \tau && \text{(Assumption)} \\
    \implies & \Gamma, z : \tau_1 \vdash v : \tau && \text{(Weakening)} \\
    \implies & \Gamma \vdash \lambda z : \tau_1 . v : \tau_{1} \rightarrow \tau && \text{(Lam)}
    % & \Gamma, z : \tau_1 \vdash v : \tau && \text{(Weakening)} \\
    % & \Gamma, z : \tau_1 \vdash e : \tau_2 \implies \Gamma \vdash z[v/e] : \tau_2 && \text{(Inductive Hypothesis)} \\
    % % \implies & \Gamma \vdash z[v/x] : \tau && \text{(Inductive Hypothesis)} \\
    % & \Gamma, x : \tau, z : \tau_1 \vdash e : \tau_2 && \text{(Subterm)} \\
    % \implies & \Gamma, z : \tau_1, x : \tau \vdash e : \tau_2 && \text{(Permutation)} \\
  \end{align*}

  [\textsc{App}]:

  \vspace*{-1em}
  \begin{align*}
    &  \trfrac[~\textsc{App}]{\Gamma, x : \tau \vdash e_1 : \tau_1 \rightarrow \tau_2 \qquad \Gamma, x : \tau \vdash e_{2} : \tau_1}{\Gamma, x : \tau \vdash e_{1} e_{2} : \tau_2} \quad \Gamma \vdash v : \tau  && \text{(Assumptions)} \\
    & \Gamma, x : \tau \vdash e_1 : \tau_1 \rightarrow \tau_2 \qquad \Gamma, x : \tau \vdash e_{2} : \tau_1 && \text{(Subterms)} \\
    \implies & \Gamma \vdash e_1[v/x] : \tau_1 \rightarrow \tau_2 \qquad \Gamma \vdash e_2[v/x] : \tau_1 && \text{(Induction Hypothesis)} \\
    \implies & \Gamma \vdash e_{1}[v/x] e_{2}[v/x] : \tau_2 && \text{(App)}
  \end{align*}


  [\textsc{Inl, Inr}]: Similar derivation applies for \textsc{Inr}.

  \vspace*{-1em}
  \begin{align*}
    & \trfrac[]{\Gamma, x : \tau \vdash e : \tau_1}{\Gamma, x : \tau \vdash \text{ inl } e : \tau_1 + \tau_2}  \quad \Gamma \vdash v : \tau && \text{(Assumptions)} \\
    & \Gamma, x : \tau \vdash e : \tau_1 && \text{(Subterm)} \\
    \implies & \Gamma \vdash e[v/x] : \tau_1 && \text{(Inductive Hypothesis)} \\
    \implies & \Gamma \vdash \inl{e[v/x]} : \tau_1 + \tau_2 && \text{(Inl)}
  \end{align*}

  [\textsc{Let}]:

  \vspace*{-1em}
  \begin{align*}
    & \trfrac[]{\Gamma, x : \tau \vdash e_1 : \tau_1 \qquad \Gamma, x : \tau, z : \tau_1 \vdash e_2
      : \tau_2}{\Gamma, x : \tau \vdash \text{ let } z = e_1 \text{ in } e_2 : \tau_2}
      \quad \Gamma \vdash v : \tau  && \text{(Assumptions)} \\
    & \Gamma, x : \tau \vdash e_1 : \tau_1 \implies \Gamma \vdash e_1[v/x] : \tau_1 && \text{(Subterm 1, Inductive Hypothesis)} \\
    & \Gamma, x : \tau, z : \tau_1 \vdash e_2 : \tau_2 && \text{(Subterm 2)} \\
    \implies & \big(\Gamma, z : \tau_1\big), x : \tau \vdash e_2 : \tau_2 && \text{(Permutation)} \\
    \implies & \Gamma, z : \tau_1 \vdash e_2[v/x] : \tau_2 && \text{(Inductive Hypothesis)} \\
    \implies & \Gamma \vdash \text{ let } z = e_1[v/x] \text{ in } e_2[v/x] : \tau_2 && \text{(Let)}
  \end{align*}


  [\textsc{Pair}]:

  \vspace*{-1em}
  \begin{align*}
    & \trfrac[]{\Gamma, x : \tau \vdash e_1 : \tau_1 \qquad \Gamma, x : \tau \vdash e_2 : \tau_2}{\Gamma, x : \tau \vdash (e_1, e_2) : \tau_1 \times \tau_2} \quad \Gamma \vdash v : \tau && \text{(Assumptions)} \\
    & \Gamma, x : \tau \vdash e_1 : \tau_1 \qquad \Gamma, x : \tau \vdash e_2 : \tau_2 && \text{(Subterms)} \\
    \implies & \Gamma \vdash e_1[v/x] : \tau_1 \qquad \Gamma \vdash e_2[v/x] : \tau_2  && \text{(Inductive Hypothesis)} \\
    \implies & \Gamma \vdash (e_1[v/x], e_2[v/x]) : \tau_1 \times \tau_2 && \text{(Pair)}
  \end{align*}

  [\textsc{Proj 1, Proj 2}]: A similar derivation applies for Proj 2.

  \vspace*{-1em}
  \begin{align*}
    &\trfrac[]{\Gamma, x : \tau \vdash (e_1, e_2) : \tau_1 \times \tau_2}{\Gamma, x : \tau \vdash e_1 : \tau_1} \quad \Gamma \vdash v : \tau && \text{(Assumptions)} \\
    &\Gamma, x : \tau \vdash (e_1, e_2) : \tau_1 \times \tau_2 && \text{(Subterm)} \\
    \implies & \Gamma \vdash (e_1, e_2)[v/x] : \tau_1 \times \tau_2 && \text{(Inductive Hypothesis)} \\
    \implies & \Gamma \vdash (e_1[v/x], e_2[v/x]) : \tau_1 \times \tau_2 && \text{(Substitution)} \\
    \implies & \Gamma \vdash e_1[v/x] : \tau_1 && \text{(Proj 1)}
  \end{align*}


\end{proof}

\begin{theorem}[Preservation]
  If
\end{theorem}

\begin{proof}
  Admitted.
\end{proof}

\subsection{Progress}

\begin{lemma}[Canonical Forms]
  If
\end{lemma}

\begin{proof}
  Admitted.
\end{proof}

\begin{theorem}[Progress]
  If $\cdot \vdash x : \tau$ is a well typed term then $x$ is a value or there
  exists some $y$ such that $x \mapsto y$.
\end{theorem}

\begin{proof}
  Admitted.
\end{proof}

\section{Acknowledgements}

Thanks to Anton Lorenzen for his guidance on the project.

\printbibliography

\end{document}